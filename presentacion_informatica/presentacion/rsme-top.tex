\documentclass{beamer}

\usepackage{beamerthemesplit}
%\usecolortheme{dolphin}

\usepackage{amsfonts}
\usepackage{amsmath}
\usepackage{amsthm}
\usepackage{url}
%\usepackage[all]{xy}
\usepackage{graphicx}
%\usepackage{pdfsync}
%\usepackage{mdframed}

\usepackage{latexsym}
\usepackage{amssymb}
%\usepackage[cp850]{inputenc}
\usepackage{epsfig}
\usepackage{psfrag}
\usepackage{amsfonts}
%\usepackage{framed}

%  \usetheme{Madrid}

%\usepackage{times}
%\usepackage[T1]{fontenc}

\newtheorem*{mthm}{Main Theorem}
\newtheorem{sat}{Theorem}[section]		
\newtheorem{lem}[sat]{Lemma}
\newtheorem{kor}[sat]{Corollary}			\newtheorem{prop}[sat]{Proposition}
\newtheorem{bei}{Example}				\newtheorem{defi}{Definition}
\newtheorem*{defi*}{Definition}			\newtheorem*{bei*}{Example}
\newtheorem*{sat*}{Theorem}				\newtheorem*{kor*}{Corollary}
\newtheorem*{rmk*}{Remark}				\newtheorem{quest}{Question}	
\newtheorem{assumption}{Assumption}[section]	


\setbeamercolor{lieu}{fg=blue!50!red}
%\definecolor{violet}{cmyk}{0.2,1,0,0.3}
\setbeamercolor{ora}{fg=red!10!orange!90!black}
\setbeamercolor{hyp}{fg=green!50!black} 
%\definecolor{vert}{rgb}{0.2,0.7,0.1}
\setbeamercolor{bleu}{fg=blue} 
\newcommand{\bleu}{\usebeamercolor[fg]{bleu}}
\newcommand{\lieu}{\usebeamercolor[fg]{lieu}}
\newcommand{\hyp}{\usebeamercolor[fg]{hyp}}
\newcommand{\ora}{\usebeamercolor[fg]{ora}}
\newcommand{\hypitem}{\setbeamercolor{item}{use={hyp},fg=hyp.fg}}	
\setbeamercolor{red}{fg=red} 
\newcommand{\red}{\usebeamercolor[fg]{red}}



\newenvironment{ftheo}
  {\begin{mdframed}\begin{theorem}}
  {\end{theorem}\end{mdframed}}


% number equations by section:
\renewcommand{\theequation}{\thesection.\arabic{equation}}
\let\ssection=\section
\renewcommand{\section}{\setcounter{equation}{0}\ssection}

\newtheorem*{namedtheorem}{\theoremname}
\newcommand{\theoremname}{testing}
\newenvironment{named}[1]{\renewcommand{\theoremname}{#1}\begin{namedtheorem}}{\end{namedtheorem}}

\theoremstyle{remark}
\newtheorem*{bem}{Remark}

%\setlength{\parindent}{0em}

%\newcommand{\defin}{\ensuremath{\overset{ \text{\tiny def} }{=} } }

\newcommand{\BC}{\mathbb C}			\newcommand{\BH}{\mathbb H}
\newcommand{\BR}{\mathbb R}			\newcommand{\BD}{\mathbb D}
\newcommand{\BN}{\mathbb N}			\newcommand{\BQ}{\mathbb Q}
\newcommand{\BS}{\mathbb S}			\newcommand{\BZ}{\mathbb Z}
\newcommand{\BF}{\mathbb F}				\newcommand{\BT}{\mathbb T}
\newcommand{\BM}{\mathbb M}			\newcommand{\BG}{\mathbb G}	
\newcommand{\BP}{\mathbb P}			

%\newcommand{\RP}{\mathrm P}

\newcommand{\CA}{\mathcal A}		\newcommand{\CB}{\mathcal B}
\newcommand{\CC}{\mathfrak C}		\newcommand{\calD}{\mathcal D}
\newcommand{\CE}{\mathcal E}		\newcommand{\CF}{\mathcal F}
\newcommand{\CG}{\mathcal G}		\newcommand{\CH}{\mathcal H}
\newcommand{\CI}{\mathcal I}			\newcommand{\CJ}{\mathcal J}
\newcommand{\CK}{\mathcal K}		\newcommand{\CL}{\mathcal L}
\newcommand{\CM}{\mathcal M}		\newcommand{\CN}{\mathcal N}
\newcommand{\CO}{\mathcal O}		\newcommand{\CP}{\mathcal P}
\newcommand{\CQ}{\mathcal Q}		\newcommand{\CR}{\mathcal R}
\newcommand{\CS}{\mathcal S}		\newcommand{\CT}{\mathcal T}
\newcommand{\CU}{\mathcal U}		\newcommand{\CV}{\mathcal V}
\newcommand{\CW}{\mathcal W}		\newcommand{\CX}{\mathcal X}
\newcommand{\CY}{\mathcal Y}		\newcommand{\CZ}{\mathcal Z}

\newcommand{\FO}{\mathfrak o}
\newcommand{\FP}{\mathfrak p}
\newcommand{\FG}{\mathfrak g}
\newcommand{\FK}{\mathfrak k}


\newcommand{\actson}{\curvearrowright}
\newcommand{\D}{\partial}
\newcommand{\DD}{\nabla}
\newcommand{\bs}{\backslash}


\DeclareMathOperator{\Out}{Out}		%	Aeussere Automorphismen einer Gruppe
\DeclareMathOperator{\Diff}{Diff}	%	Diffeomorphimen einer Mf
\DeclareMathOperator{\SL}{SL}		%	Spezielle lineare Gruppe
\DeclareMathOperator{\PSL}{PSL}		%	Spezielle lineare Gruppe
\DeclareMathOperator{\GL}{GL}		%	Allgemeine lineare Gruppe
\DeclareMathOperator{\Id}{Id}		%	Identit\"at
\DeclareMathOperator{\Isom}{Isom}	%	Isometrien einer Mf
\DeclareMathOperator{\Hom}{Hom}		%	Homomorphismen
\DeclareMathOperator{\vol}{vol}		%	Volumen
\DeclareMathOperator{\tr}{Tr}
\DeclareMathOperator{\Map}{Map}
\DeclareMathOperator{\Mod}{Mod}
\DeclareMathOperator{\PMod}{PMod}

\DeclareMathOperator{\inj}{inj}
\DeclareMathOperator{\diam}{diam}
\DeclareMathOperator{\rank}{rank}
\DeclareMathOperator{\Axis}{Axis}
\DeclareMathOperator{\maxrad}{maxrad}
\DeclareMathOperator{\height}{height}
\DeclareMathOperator{\rel}{rel}
\DeclareMathOperator{\arccosh}{arccosh}
\DeclareMathOperator{\Ker}{Ker}
\DeclareMathOperator{\End}{End}
\DeclareMathOperator{\length}{length}
\DeclareMathOperator{\diver}{div}
\newcommand{\comment}[1]{}
\DeclareMathOperator{\waist}{waist}
\newcommand{\x}{{\bf x}}
\DeclareMathOperator{\SO}{SO}
\DeclareMathOperator{\OO}{O}
\DeclareMathOperator{\Gal}{Gal}
\DeclareMathOperator{\ext}{ext}
\DeclareMathOperator{\Tor}{Tor}
\DeclareMathOperator{\Comm}{Comm}
\DeclareMathOperator{\Domain}{Domain}
\DeclareMathOperator{\Aut}{Aut}
%\DeclareMathOperator{\Stab}{Stab}
%\DeclareMathOperator{\Fix}{Fix}
\DeclareMathOperator{\inte}{int}
%\DeclareMathOperator{\red}{{red}}
\DeclareMathOperator{\PGL}{PGL}
\DeclareMathOperator{\Homeo}{Homeo}

\DeclareMathOperator{\vcd}{{vcd}}
\DeclareMathOperator{\Ad}{Ad}
\DeclareMathOperator{\cdim}{cd}
\DeclareMathOperator{\sing}{sing}
\DeclareMathOperator{\Sp}{Sp}
\DeclareMathOperator{\UU}{U}
\DeclareMathOperator{\gdim}{\underline{gd}}
\DeclareMathOperator{\cdm}{\underline{cd}}
\DeclareMathOperator{\SU}{SU}
\DeclareMathOperator{\Gr}{Gr}
\DeclareMathOperator{\Span}{Span}
\DeclareMathOperator{\Teich}{Teich}

\def\XX{\mathfrak X}
\def\BB{\mathfrak B}



\mode<presentation>

\title[Universal mapping class groups]{Universal mapping class groups}
\author{Javier Aramayona}
\institute[UAM/ICMAT] {
 UAM/ICMAT\\
%\vspace{10pt}
\begin{center}
\footnotesize{{\bf {Joint work with Louis Funar (Grenoble)}}}
\end{center}}
\date{}






\begin{document}

\maketitle


%-------------------------------------------------------------------------------


\frame{\frametitle{Mapping class group}

{ $S$: orientable surface, of finite or infinite topological type.
\bigskip
\begin{itemize}
\item $\Mod(S)$: the group of isotopy classes of homeos $S\to S$. 

\medskip
\item $\PMod(S)$: the {\em pure} subgroup (elements pointwise fix boundary and punctures).
\end{itemize}
 
}

\medskip

{\begin{example} $\Mod(\rm{torus}) = \PMod(\rm{torus})=  \GL(2,\mathbb Z)$.\end{example}
}

}

%---------------------------------------

\frame{\frametitle{Finite presentation}

{ 

\begin{sat}[Dehn-Lickorish, Hatcher-Thurston, etc]
If $S$ has finite type, then $\Mod(S)$ is finitely presented. In fact, $\PMod(S)$ is generated by a finite collection of Dehn twists. 
\end{sat}
}


\medskip


\begin{proof}
Construct a nice action of $\Mod(S)$ on a simply-connected complex (built from simple closed curves on $S$). 
\end{proof}

}

%---------------------------------------

\frame{\frametitle{Universal mapping class groups}

{Groups that contain {\em every} mapping class group (of genus $g$). 

\bigskip

\noindent{\bf Motivation}

\begin{itemize}
\item Homological stability
\item Dynamics
\end{itemize}



}

}

%---------------------------------------

\frame{\frametitle{Homological stability I}

{
 $S_{g,n}$: surface of genus $g$ with $n$ boundary components. Observe that $S_{g,n} \hookrightarrow S_{g,n+1}$ for all $n$ (glue a pair of pants). Can form \[\lim_{\to_n} \PMod(S_{g,n})\]

\begin{rmk*}
\begin{itemize}
\item $\lim_{\to_n} \PMod(S_{g,n})$ contains $\PMod(S_{g,n})$ for all $n\ge 1$. 
\item It is infinitely generated. 
\end{itemize}
\end{rmk*}

}}


%---------------------------------------



\frame{\frametitle{Homological stability II}

{

\bigskip
\begin{sat}[Harer's stability theorem, 1985]
If $g>>i$, then \[H_i(\PMod(S_{g,n}),\BZ) \cong H_i(\lim_{\to_k} \PMod(S_{g,k}),\BZ)\]
\end{sat}

(i.e. $H_i$ does not depend on number of boundary components)

}

}

%---------------------------------------

\frame{\frametitle{Big mapping class groups}

{Notation: $\Sigma_g$ is  $S_{g,0}$ minus a Cantor set. Consider $\Mod(\Sigma_g)$. 

%\begin{figure}[htb]
%\begin{center}
%\includegraphics[width=3in,height=2in]{surfinf0.pdf} \caption{A %compact piece of $\Sigma_0$.} \label{fig:genus0}
%\end{center}
%\end{figure}

}}

%---------------------------------------

\frame{\frametitle{Big mapping class groups II}

{
\begin{rmk*}
\begin{itemize}
\item Appear naturally in dynamics (D. Calegari)
\medskip
\item $\Mod(\Sigma_g)$ contains $\PMod(S_{g,n})$ for all $n\ge 1$.
\medskip
\item $\Mod(\Sigma_g)$ is uncountable! 
\medskip
\item  ${\lim_{\to_k}} \PMod(S_{g,k}) = \PMod_c(\Sigma_g)$
\end{itemize}
\end{rmk*}






}

}

%---------------------------------------

\frame{\frametitle{Asymptotically rigid homeos}

{Defined in terms of a {\em rigid structure} on $\Sigma_g$: a pants decomposition $P$, plus some other things. (Point: if $f$ fixes rigid structure then $f= id$.)

\bigskip

\begin{definition}
A homeomorphism $f:\Sigma_g \to \Sigma_g$ is {\em asymptotically rigid} if there exists a $P$-suited compact subsurface $Z$ with $f(Z)$ $P$-suited, such that $f$ preserves the rigid structure outside these. 
\end{definition}


}

}


%---------------------------------------

\frame{\frametitle{Asymptotic mapping class group}

{


\begin{definition}
$B_g$: the group of isotopy classes of asymptotically rigid homeomorphisms.
\end{definition}

\medskip

\begin{sat}[A-Funar]
One has \[1 \to P\Mod_c(\Sigma_g) \to B_g \to V \to 1\] where $V$ is Thompson's group $V$. 
\end{sat}

\medskip

($V$ is a simple, infinite, finitely-presented group with lots of other cool properties. It's a subgroup of $\Homeo(\rm{Cantor})$)


}

}

%---------------------------------------

\frame{\frametitle{Finite presentation}


{

\begin{sat}[A-Funar]
For all $g$, the group $B_g$ is finitely presented. 
\end{sat}

(Known by Funar-Kapoudjian (2004) for $g=0$.) 

\medskip


\begin{proof}
Construct a nice action on a simply-connected complex (built from simple closed curves on $\Sigma_g$). 
\end{proof}


}

}

%---------------------------------------

\frame{\frametitle{Homology}


{

\begin{sat}[A-Funar]
For $g>>i$, $H_i(B_g, \BZ)$ is the stable homology group of the m.c.g of genus $g$. (Corollary: $B_g$ is perfect for $g\ge 3$.)
\end{sat}

\medskip

\begin{proof}
Promote Harer's stability  to $B_g$ using the short exact sequence above. 
\end{proof}




}

}

%---------------------------------------

\frame{\frametitle{Injections}


{

\begin{sat}[A-Funar]
If $h>g$ then no (weakly) injective homomorphisms $B_h \to B_g$. (Corollary: Class. of the $B_g$'s up to isomorphism.)
\end{sat}

\medskip

\begin{proof}
Use properties of $V$, plus the analogous statement for $\PMod(S_{g,n})$ (Castel, A-Souto). 
\end{proof}


}

}

%---------------------------------------

\frame{\frametitle{Algebraic rigidity}


{

\begin{sat}[Ivanov]
If $S$ has finite-type then \[\Aut(\Mod(S)) = \Aut(\PMod(S))= \Mod(S).\]
\end{sat}

\medskip

\begin{sat}[A-Funar]
For every $g$, $\Aut(B_g) = N_{\Mod(\Sigma_g)}(B_g)$. 
\end{sat}

\begin{proof}
Prove that Dehn twists go to Dehn twists. 
\end{proof}

%\bigskip
%
%(Analogs of similar theorems for finite-type MCGs.)


}

}


%---------------------------------------

\frame{\frametitle{Homomorphisms from lattices}


{

\begin{sat}[Farb-Masur]
Let $\Gamma$ be an higher-rank lattice, $S$ a surface of finite type.  Then every homomorphism $\Gamma \to \Mod(S)$ has finite image. 
\end{sat}

\medskip

\begin{sat}[A-Funar]
Let $\Gamma$ be an higher-rank lattice. Then every homomorphism $\Gamma \to B_g$ has finite image. 
\end{sat}

\begin{proof}
Use properties of $V$ plus the analogous statement for $\PMod(S_{g,n})$ (Farb-Masur...) 
\end{proof}




}

}

%---------------------------------------

\frame{\frametitle{Kazhdan's Property (T)}

{
\begin{quest}
Do finite-type mapping class groups have Property (T)? 
\end{quest}

(The expected answer is ``no" (Andersen).) 

\bigskip

\begin{sat}[A-Funar]
$B_g$ does not have Property (T).
\end{sat}

\begin{proof}
$B_g$ surjects onto $V$, which is infinite and has the {\em Haagerup property}. 
\end{proof}



}

}

%---------------------------------------

\frame{\frametitle{Linearity}


{ 

\begin{quest}
Are mapping class groups linear?
\end{quest}

\medskip

Known only for the closed surface of genus 2 (Bigelow-Budney). 

\bigskip

\begin{sat}[A-Funar]
$B_g$ is not linear. 
\end{sat}

\begin{proof}
$B_g$ contains a copy of Thompson's group $F$, which is not linear. 
\end{proof}



}

}

%---------------------------------------

\frame{\frametitle{A related group}


{ $H_g$ (contains $B_g$) consists of asymptotically rigid homeos, in a weaker sense. 

\bigskip



\begin{sat}[A-Funar]
$H_g$ is finitely presented, and is dense in $\Mod(\Sigma_g)$. 
\end{sat}

\begin{proof}
Given a mapping class, construct {\em by hand} a sequence in $B_g$ that converges to it. 
\end{proof}



}

}




\end{document}
 